\lecture{Регистры сдвига и линейная сложность}{А. Рязанов}{И. Агафонова}

\section{Регистры сдвига с линейной обратной связью}

Хотим генерировать поток битов из некоторого начального конечного количества.
Рассмотрим следующий алгоритм:

\begin{algorithm}
Имеем $s_0, s_1, \ldots, s_{l-1}$, где $l$ будем называть длиной регистра
сдвига. Пусть $f: \tw^l \to \tw$. Тогда будем генерировать дальнейшие биты
по рекуррентному соотношению $s_i = f(s_{i-1}, s_{i-2}, \ldots, s_{i-l})$.

$f$ будем называть функцией обратной связи.\\
$s_i$ --- выход регистра на шаге $i$.
\end{algorithm}

Рассмотрим регистр сдвига с линейной функцией обратной связи (РСЛОС)

\begin{definition}
Если $f(x_0, \ldots, x_{l-1}) = \sum\limits_{i=0}^{l-1} c_i x_i$, то
многочлен, ассоциированный с РСЛОС: $c(x) = 1 + c_0 x + \ldots + c_{l-1} x^l$.
\end{definition}

\begin{definition}
Периодом регистра называется число
$\min \{N \in \mathbb{N} \colon \forall i \ge N \, S_{N+i} = S_i\}$
\end{definition}

{\bfseries Свойства:}
\begin{enumerate}
\item $s_0 = \ldots = s_{l-1} = 0 \implies \forall i \colon s_i = 0$
\item Период регистра конечен.
\begin{proof}
Если $$
\left\{\begin{array}{rl} s_i &= s_j \\
                         s_{i+1} &= s_{j+1} \\
                         \ldots & \ldots \\
                         s_{i+l-1} &= s_{j+l-1} 
        \end{array}\right.$$
   То $s_{i+l} = s_{j+l}$ по определению. Тогда $\forall k \ge 0 \colon s_{i+k} = s{j+k}$.
   Таким образом $(s_i, s_{i+1}, \ldots) = (s_j, s_{j+1}, \ldots)$. Но
   тогда существует не более $2^l$ различных типов таких последовательностей.

\end{proof}
\item $T \le 2^l - 1$. Непосредственно следует из доказательства предыдущего пункта.
\item $c_{l-1} = 0 \implies $ период начинается не с начала последовательности
      ($c_T$ не всегда равно $c_0$)
\item $c_{l-1} = 1 \implies c_T = c_0$
\item $c(x)$ неприводим над $\mathbb{F}_2$ $\implies$ $2^l-1$ кратно $T$
\item $c(x)$ примитивный над $\mathbb{F}_2$ $\implies$ $T = 2^l - 1$
\end{enumerate}

\begin{proposition}
Пусть известны $s_i, \ldots, s_{i+2l-1}$ и известно, что регистр имеет длину
$l$. Тогда можно найти регистр сдвига, порождающий такую последовательность.
\end{proposition}

\begin{proof}
Составим систему уравнений, относительно $c_i$:
$$\left\{\begin{array}{rcl}   
   s_{i+l} &=& c_0 s_{i+l - 1} + c_1 s_{i+l-2} + \ldots + c_{l-1} s_0\\
   s_{i+l+1} &=& c_0 s_{i+l} + c_1 s_{i+l-1} + \ldots + c_{l-1} s_1\\   
   \ldots &=& \ldots\\
   s_{i+2l-1} &=& c_0 s_{i+2l - 2} + c_1 s_{i+2l-3} + \ldots + c_{l-1} s_{i+l-1}
   
\end{array}\right.$$
Система совместна по построению $s_i$, тогда решение --- подходящий регистр
сдвига. Если уранения линейно-независимы, регистр сдвига определяется однозначно.
\end{proof}

\section{Линейная сложность, алгоритм Берлекэмпа-Мэсси}
\begin{definition}
Регистр сдвига порождает последовательность $s$, если для начальных
значений $s_0, \ldots, s_{l-1}$ регистр выдает последовательность $s$.
\end{definition}

\begin{definition}
Линейной сложностью последовательности бит (конечной или 
бесконечной) $s$ назовем
\begin{itemize}
\item $0$, если $s = (0, 0, \ldots)$
\item $\infty$, если $\not\exists$ РСЛОС, порождающего $s$.
\item Длина минимального регистра сдвига, порождающего $s$.
\end{itemize}

Обозначим $L(s)$.
\end{definition}

\begin{definition}
Пусть $s$ --- последовательность бит. Тогда пусть $$L_N = L(s_0, \ldots, s_{N-1})$$
Последовательность $L_1, L_2, \ldots$ назовем профилем линейной сложности
последовательности $s$.
\end{definition}


\begin{proposition}
Верны следующие утверждения
\begin{enumerate}
\item $j > i \implies L_j \ge L_i$
\item $L_N \le {N \over 2} \implies L_{N+1} > L_N$
\item $L_{N+1} > L_N \implies L_{N} + L_{N+1} = N+1$
\end{enumerate}
\end{proposition}

Сам алгоритм базируется на этих трех утверждениях. \emph{[Можно его дописать
сюда].}

\section{Порождение симплексного кода с помощью регистра сдвига}

\begin{definition}
Рассмотрим $C_m$, $(2^m - 1, 2^m - m - 1)$-код Хэмминга. Дуальный к нему код
$S_m$ является кодом Адамара с матрицей Сильвестра --- симплексным кодом.
\end{definition}

\begin{remark}
$S_m$ является циклическим кодом с проверочным многочленом 
$$h(x) = 1 + h_1 x + \ldots + h_{m-1} x^{m-1} + x^m$$

Тогда, вспоминая структуру проверочной матрицы циклического кода,
можем записать условия на то, что $(s_0, \ldots, s_{2^m - 2}) \in S_m$:
$$\forall i \in \{0, \ldots 2^m-2-m\} \colon
  s_{i+m} = s_i + s_{i+1} h_{m-1} + \ldots + s_{i+m-1} h_1$$
\end{remark}

Тогда каждое кодовое слово $s \in S_m$ порождается регистром сдвига с характеристическим
многочленом $h(x)$ и начальными входами $s_0, \ldots, s_{m-1}$.


