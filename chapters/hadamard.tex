\lecture{Матрицы и коды Адамара}{А. Рязанов}{И. Агафонова}

\section{Матрицы и коды Адамара, общее представление}

\begin{definition}
Матрицей Адамара называется матрица $H \in \{-1,1\}^{n \times n}$, такая,
что $H \cdot H^{T} = n E_n$.

Матрица адамана в нормализованном виде --- это матрица, у которой первая
строка и первый столбец состоят из единиц.

Двоичная матрица Адамара, это матрица, полученная из матрицы Адамара 
заменой $-1$ на $1$ а $1$ на $0$.
\end{definition}

\begin{proposition}
Умножение строчки или столбца матрицы Адамара на $-1$ переводит
ее в матрицу Адамара.
\end{proposition}

\begin{proof}
 Умножение строчки или столбца на единицу, это доножение слева или
 справа на матрицу $d = diag(1,\ldots,1, -1, 1, \ldots, 1)$.
 Тогда в первом случае 
  $$(d H) \cdot (d H)^{T} = d H H^{T} d^T = d (n E) d^{T} = n E d d^{T} = n E$$
 а во стором
 $$(H d) \cdot (H d)^{T} = H d d^{T} H^{T} = H H^{T} = n E$$ 
\end{proof}

\begin{theorem}
Если существует матрица Адамара порядка $n$, то $n \in \{1,2\} \cup \{4k\}$
\end{theorem}

\begin{proof}
Пусть $n \ge 3$ и существует $H$. Тогда представим ее в нормализованном виде
и разделим столбцы на четыре типа:
\begin{enumerate}
\item Начинается с $(1,1,1)$ --- $i$ штук
\item Начинается с $(1,1,-1)$ --- $j$ штук
\item Начинается с $(1,-1,1)$ --- $k$ штук
\item Начинается с $(1,-1,-1)$ --- $l$ штук
\end{enumerate}
Запишем условия ортогональности строк $(1,2)$, $(2,3)$ и $(1,3)$:
$$\left\{\begin{array}{l}
           i+j -k-l = 0 \\
           i-j + k -l = 0 \\
           i-j-k+l = 0
         \end{array}\right.$$
Тогда $i=j=k=l$, тогда $n = 4i$         
\end{proof}

\begin{proposition}
Если $H$ --- матрица Адамара, то
$$\begin{pmatrix} H & H \\ H & -H \end{pmatrix}$$ --- тоже матрица Адамара.
\end{proposition}
\begin{proof}
$$\begin{pmatrix} H & H \\ H & -H \end{pmatrix} \cdot
   \begin{pmatrix} H^{T} & H^{T} \\ H^{T} & -H^{T} \end{pmatrix}
   = \begin{pmatrix} H H^{T} + H H^{T} & H H^{T} - H H^{T} \\
                      H H^{T} - H H^{T} & H H^{T} + H H^{T}\end{pmatrix} = 2n E_{2n}$$
\end{proof}

Такие матрицы Адамара называются матрицами Сильвестра.

\begin{definition}
Симплексным кодом Адамара называется $[K-1, K, {K \over 2}]$-код, состоящий
из строк двоичной матрицы Адамара из которой удален первый столбец.
\end{definition}

\begin{proposition}
Для симплексного кода Адамара выполнено $K = {2d \over 2d - n}$.
\end{proposition}
\begin{proof}
Очевидно.
\end{proof}

\begin{remark}
Если матрица Адамара, построена по способу Сильвестра, то симплексный код, 
построенный по ней, линеен.
\end{remark}

\section{Построение матрицы Адамара по способу Пэли}

\begin{definition}
Пусть $p \in \mathbb{P} \setminus \{2\}$. $\{a \in \{0,\ldots, p-1\}
  \colon \exists b \colon b^2 = a \}$ называется множеством
  квадратичных вычетов.
\end{definition}

\begin{definition}
Функция
$$\chi(i) = \begin{cases} 0 & i \text{ кратно } p \\
                          1 & i \mod p \text{ вычет} \\
                          -1 & i \mod p \text{ невычет}
            \end{cases}$$
называется символом Лежандра.
\end{definition}

\begin{theorem}
$\forall c \neq 0 \mod p$ выполнено $\sum\limits_{b=0}^{p-1} \chi(b) \chi(b+c) = -1$
\end{theorem}

\begin{construction}
Матрица Джекобстола. $Q = \{q_{ij}\}_{p \times p}$. $q_{ij} = \chi(j-i)$.
\end{construction}

\begin{lemma}
$Q \cdot Q^{T} = p E - \mathbf{1}_{p \times p}$ 

$Q \mathbf{1}_{p \times p} = \mathbf{1}_{p \times p} Q = 0$
\end{lemma}
\begin{proof}
$Q \mathbf{1}_{p \times p} = \mathbf{1}_{p \times p} Q = 0$, так как по модулю
$p$ существует ${p-1 \over 2}$ вычетов и ${p-1 \over 2}$ невычетов.

Рассмотрим $P = \{p_{ij}\} = Q \cdot Q^{T}$. Тогда 
$$\begin{array}{rl} p_{ii} &= 
    \sum\limits_{k=0}^{p-1} q_{ik}^2 = p\\
    p_{ij} &= \sum\limits_{k=0}^{p-1} q_{ik} q_{jk} \end{array}$$
$$p_{ij} = \sum\limits_{k=0}^{p-1} \chi(i-k) \chi(j-k) =
   \sum\limits_{k=0}^{p-1} \chi(i-k) + \chi ((i-k) + (j-i)) = -1$$
\end{proof}

\begin{lemma}
Пусть
$$H = \begin{pmatrix} 1 & \mathbf{1}_p \\ \mathbf{1}_p & Q - E \end{pmatrix}$$
Тогда $H$ --- матрица Адамара
\end{lemma}

\begin{proof}
$$H \cdot H^{T} = 
\begin{pmatrix} 1 & \mathbf{1}_p \\ \mathbf{1}_p & Q - E \end{pmatrix} \cdot
\begin{pmatrix} 1 & \mathbf{1}_p \\ \mathbf{1}_p & Q^{T} - E \end{pmatrix} =
\begin{pmatrix} p+1 & \mathbf{0}_p \\ \mathbf{0}_p & 
\mathbf{1}_{p \times p} + (Q-E)(Q^{T}-E) \end{pmatrix} $$
Распишем
$$\mathbf{1}_{p\times p} + (Q- E) (Q^{T}-E) = 
\mathbf{1}_{p\times p} + Q Q^{T} -Q - Q^{T} + E
\mathbf{1}_{p\times p} + Q Q^{T} -Q - Q^{T} + E$$
заметим, что $q_{ij} = \chi(i-j) = \chi(-1)\chi(j-i) = -\chi(j-i)$, тогда 
$Q^{T} = -Q^{T}$, тогда 
$$\mathbf{1}_{p\times p} + Q Q^{T} -Q - Q^{T} + E = 
  \mathbf{1}_{p\times p} + Q Q^{T} + E = (p+1) E$$
\end{proof}

