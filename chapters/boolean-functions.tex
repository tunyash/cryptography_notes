\lecture{Булевы функции}{}{И. Агафонова}

\section{Определения. Алгебраическая нормальная форма}
\subsection{Алгебраическая нормальная форма}
\begin{definition}
Функция $f: \{0,1\}^m \to \{0,1\}$ называется булевой функцией
\end{definition}

Мы поймем, что \emph{любую} булеву функцию можно представить в виде 
многочлена от $m$ переменных в $\mathbb{F}_2$ и даже выведем явную формулу. 

С помощью таблицы истинности можно легко заметить, что для любой булевой
функции $f$ верно

$$f(v_1, \ldots, v_m) = \bigvee\limits_{i_1, \ldots, i_m \in \{0,1\}}
       f(i_1, \ldots, i_m) (w_1^{i_1} \land \ldots \land w_m^{i_m})$$

Где $w_i^1 = v_i$ и $w_i^0 = \lnot v_i$.

Это просто функция в дизъюнктивной нормальной форме.

\begin{definition}
Будем обозначать $x \le y$ для $x,y \in \{0,1\}^n$, если $x_i \le y_i$ для 
всех $i \in \{1,\ldots,n\}$
\end{definition}

\begin{theorem}
\label{mebius}
Любая булева функция $f$ может быть записана как
$$f(v_1, \ldots, v_m) = \sum\limits_{a \in \{0,1\}^m} g(a) v_1^{a_1}
 \ldots, v_m^{a_m}$$ где $g(a) = \sum\limits_{b \le a} f(b_1, \ldots, b_m)$

{\bfseries Здесь и далее, если не указано иное, все суммы в $\mathbb{F}_2$}
\end{theorem}

\begin{proof}
Зафиксируем набор значений $v_1, \ldots, v_m$ и проверим, что получается то,
 что нужно. Пусть $A = \{i \colon v_i = 1\} = \{\alpha_1, \ldots, \alpha_k\};
 \, B = \{i \colon v_i = 0\}$.

Во-первых, можно выбросить слагаемые, где $a_i = 1$ для $i \in B$, так 
как они обращаются в ноль.

$$\sum\limits_{a \le \mathbf{1}_A} g(a) v_1^{a_1} \ldots, v_m^{a_m} = 
\sum\limits_{a \le \mathbf{1}_A} g(a)$$

{\scriptsize Здесь $\mathbf{1}_A$ --- характеристический вектор $A$, он
 равен $(v_1, \ldots, v_m)$.}

Так как во всех остальных слагаемых $v_1^{a_1} \ldots, v_m^{a_m} = 1$.
 Тогда, подставляя $g(a)$, получаем
$$=  \sum\limits_{a \le \mathbf{1}_A} \sum\limits_{b \le a} f(b_1, \ldots, b_m)$$
Давайте поймем, сколько раз каждое слагаемое входит в сумму:
$$ = \sum\limits_{b \le \mathbf{1}_A} \sum\limits_{\mathbf{1}_A \ge a \ge b}
 f(b_1, \ldots, b_m)$$
Осталость посчитать сколько бывает таких $a$. Легко видеть, что их
 количество равно $2^{w(a) - w(b)}$ и тогда все слагаемые, кроме 
 $b = \mathbf{1}_A = (v_1, \ldots v_m)$ по четности обращаются в ноль
$$ = f(v_1, v_2, \ldots, v_m)$$
\end{proof}



\begin{definition}
Представление
$$f(v_1, \ldots, v_m) = \sum\limits_{a \in \{0,1\}^m} g(a) v_1^{a_1}
 \ldots, v_m^{a_m}$$
 называется \emph{алгебраической нормальной формой} функции $f$.
\end{definition}

\subsection{Быстрое преобразование Мёбиуса}
\begin{definition}
Пусть $f \in \tw^{2^n}$. Поставим вектору $f$ в соответствие будеву функцию
$\in \map(\tw^n, \tw)$ следующим образом
$$f \mapsto (\underbrace{x}_{\in \tw^n} \mapsto f_{x})$$
где $f_x$ --- компонента вектора $f$ с номером, соответствующим двоичной
записи $x$. Мы будем отождествлять вектор $f$ и соответствующую
ему функцию и записывать $f(x) = f_x$.

Отображение $\mu: \tw^{2^n} \to \tw^{2^n}$ называется 
\emph{преобразованием Мёбиуса}, если выполнено:
   $$f \mapsto g \iff \forall a \colon g(a) = \bigoplus\limits_{b \le a} f(b)$$
\end{definition}

Вычисление преобразования Мёбиуса по определению требует $3^n$ операций
(это количество пар $x, y \in \tw^n \colon x \le y$). Но можно выполнить
его оптимальнее, используя $2^n \cdot n$ операций.

Действительно, рассмотрим $f$ и $g = \mu(f)$. 
$$g(a_1, \ldots, a_n) = \bigoplus\limits_{b \le a} f(b_1, \ldots, b_n) = 
\bigoplus\limits_{\substack{b \le a \\ b_1 = 0}} f(b_1, \ldots, b_n) \oplus
\bigoplus\limits_{\substack{b \le a \\ b_1 = 1}} f(b_1, \ldots, b_n)$$

Теперь разберем два случая: $a_1 = 0$ и $a_1 = 1$:
$$g(0, a_2, \ldots, a_n) = \bigoplus\limits_{\substack{b \le a \\ b_1 = 0}} f(b_1, \ldots, b_n)$$
в этом случае второе слагаемое обращается в ноль, поскольку 
$b_1 = 1 \implies b \not\le a$.
$$g(1, a_2, \ldots, a_n) = 
\bigoplus\limits_{\substack{b \le a \\ b_1 = 0}} f(b_1, \ldots, b_n) \oplus
\bigoplus\limits_{\substack{b \le a \\ b_1 = 1}} f(b_1, \ldots, b_n)$$
Заметим, что теперь в обоих случаях условие 
$b \le a \iff (b_2, \ldots, b_n) \le (a_2, \ldots, a_n)$. Это дает нам
возможность рассмотреть функции $f_0 (a') = f(0, a'_1, \ldots, a'_{n-1})$
и $f_1(a') = f(0, a'_1, \ldots, a'_{n-1})$ и, используя прошлые рассуждения,
записать
$$g(0, a_2, \ldots, a_n) = \mu(f_0) (a_2, \ldots, a_n)$$
и 
$$g(1, a_2, \ldots, a_n) = \mu(f_0) (a_2, \ldots, a_n) \oplus
                           \mu(f_1) (a_2, \ldots, a_n))$$
Таким образом, мы свели задачу нахождения преобразования
Мёбиуса для вектора из $\tw^{2^n}$ к двум задачам нахождения преобразования
Мёбиуса для вектора из $\tw^{2^{n-1}}$ тогда время работы нашего
алгоритма равно $T(n) = 2^n + 2T(n-1)$. Из этого соотношения легко
видеть, что $T(n) = 2^n \cdot n$.      

\section{Коды Рида-Маллера}                    
                               

\begin{definition}
Для произвольного $r \in \{0, \ldots, m\}$ двоичный код Рида-Маллера $\mathscr{R}(r,m)$ 
порядка $r$ и длины $2^m$ определяется как 
    $$Lin \{v_{\alpha_1} \cdot \ldots \cdot v_{\alpha_p} \colon p \le r; \, 1 \le \alpha_i \le m\}$$
то есть линейная оболочка мономов степени $\le r$ или, что то же самое, множество всех многочленов от $m$ 
переменных над $\mathbb{F}_2$ степени не больше $r$.

Собственно кодами будут характеристические векторы этих многочленов.
\end{definition}

Очевидно, что этот код является линейным. Значит, можно говорить о его размерности.

\begin{note}
Размерность $\mathscr{R}(r,m)$ равна $\sum\limits_{k=0}^r C_m^k$
\end{note}

\begin{proof}
Из теоремы \ref{mebius} все мономы линейно независимы, а количество мономов степени $\le r$ равно
$\sum\limits_{k=0}^r C_m^k$
\end{proof}

\subsection{Взаимосвязь кодов Рида-Маллера разных порядков}

\begin{theorem}
$$\mathscr{R}(r+1,m+1) = \{|u|u+v| : u \in \mathscr{R}(r+1,m),\, v \in \mathscr{R}(r,m)\}$$
Здесь $|x|y|$ --- конкатенация $x$ и $y$
\end{theorem}

\begin{lemma}
Пусть $f(v_1, \ldots, v_m)$ --- булева функция с характеристическим вектором $\phi$. Тогда
характеристические векторы функций $g(v_1, \ldots, v_{m+1}) = f(v_1, \ldots, v_m)$ и $h(v_1, \ldots, v_{m+1}) = v_{m+1} f(v_1, \ldots, v_m)$
равны, соответственно $|\phi|\phi|$ и $|\mathbf{0}|\phi|$
\end{lemma}

\begin{proof}
Здесь $m+1$ cчитается старшей степенью. Тогда левой части соответствуют те значения переменных
где $v_{m+1} = 0$, а правой --- те, где $v_{m+1}=$. Тогда в обеих частях характеристического 
вектора $f$ будет вектор $\phi$. Левой части характеристического вектора $h$ будет соответствовать
тождественный $0$, поскольку мы умножили на 0.
\end{proof}

\begin{proof}

Рассмотрим $f \in \mathscr{R}(r+1,m+1)$. Давайте запишем 
   $$f(v_1, \ldots, v_{m+1}) = \underbrace{g(v_1, \ldots, v_m)}_{deg \le r+1} + 
   v_{m+1} \underbrace{h(v_1, \ldots, v_m)}_{deg \le r}$$
  Вспомним лемму и заметим, что характеристические векторы слагаемых этой формулы равны 
  $|\mathbf{1}_g|\mathbf{1}_g|$ и $|\mathbf{0}|\mathbf{1}_h|$ соответственно. Тогда
  характеристический вектор их суммы равен $|\mathbf{1}_g|\mathbf{1}_g+\mathbf{1}_h|$.
\end{proof}

\begin{note}
Похоже на формулу для биномиальных коэффициентов.
\end{note}

Теперь мы готовы к тому, чтобы найти расстояние между кодовыми словами в коде Рида-Маллера.

\begin{theorem}
Минимальное расстояние между словами в коде $\mathscr{R}(r,m)$ равно $2^{m-r}$
\end{theorem}

\begin{proof}
Индукция по $m$. При $m=0$ существует один код Рида Миллера: $\mathscr{R}(0,0)$. Он 
состоит из слов $0$ и $1$, расстояние между ними равно $1$.

Пусть для всех $m < m_0$ доказано, докажем для $m_0$. Из прошлой теоремы $\mathscr{R}(r,m) = 
\mathscr{R}(r,m-1) + \mathscr{R}(r-1,m-1)$. Рассмотрим $a_1, a_2 \in \mathscr{R}(r,m)$. Они имеют
вид $|u_1|u_1+v_1|$ и $|u_2|u_2+v_2|$ соответственно. Тогда 
$$d(a_1,a_2) = d(u_1, u_2) + d(u_1 + v_1, u_2 + v_2) \ge d(u_1,u_2) +
 |\underbrace{d(u_1,u_2)}_{\ge 2^{m-r-1}} - \underbrace{d(v_1,v_2)}_{\ge 2^{m-r}}|$$
 
 Поймём, что это неравенство действительно верно (здесь сумма вещественных чисел):
 $$d(u_1 + v_1, u_2 + v_2) = \sum\limits_{i=1}^{2^m} d_i(u_1 + v_1, u_2 + v_2)$$
 где $d_i(x,y) = 1$, если $i$-е символы $x$ и $y$ различаются и $0$, если совпадают.
 Теперь разбором случаев можно доказать, что $d_i(a+b,c+d) \le |d_i(a,c) - d_i(b,d)|$:
 \begin{center}
 \begin{tabular}{| c c c c | c c c | c |}
 \hline
 a & b & c & d & $d_i(a+b,c+d)$ & $d_i(a,c)$ & $d_i(b,d)$ & $|\ldots|$ \\
 \hline
 0 & 0 & 0 & 0 & 0 & 0 & 0 & 0 \\
 0 & 0 & 0 & 1 & 1 & 0 & 1 & 1 \\
 0 & 0 & 1 & 0 & 1 & 1 & 0 & 1 \\
 0 & 0 & 1 & 1 & 0 & 1 & 1 & 0 \\
 \hline
 \end{tabular}
 \end{center}

Здесь можно считать $a=b=0$, так как иначе можно перейти к $a \to a+a;\, b \to b+b;\, c \to c+a;\, d \to d + b$,
не изменив обе части формулы и обратив $a,b$ в ноль. Таким образом можем записать

$$\sum\limits_{i=1}^{2^m} d_i(u_1 + v_1, u_2 + v_2) \le \sum\limits_{i=1}^{2^m} |d_i(u_1,u_2) - d_i(v_1,v_2)| \le
  |\sum\limits_{i=1}^{2^m} \big(d_i(u_1,u_2) - d_i(v_1,v_2)\big)| = |d(u_1,u_2) - d(v_1,v_2)|$$

Теперь нужно разобрать два случая:
\begin{itemize}
\item $d(u_1,u_2) \ge d(v_1, v_2)$. Тогда $d(a_1, a_2) \ge 2^{m-r} + |\ldots| \ge 2^{m-r}$
\item $d(v_1,v_2) > d(u_1, u_2)$. Тогда $d(a_1,a_2) \ge d(u_1,u_2) + d(v_1, v_2) - d(u_1,u_2) = d(v_1,v_2) \ge 2^{m-r}$
\end{itemize}
\end{proof}

\subsection{Выколотые коды Рида-Маллера}
\begin{definition}
Для произвольного $r \in \{0, \ldots, m-1\}$ {\bfseries выколотый} двоичный код Рида-Маллера 
$\mathscr{R}^*(r,m)$ порядка $r$ и длины $2^m$ определяется как 
    $$\{x_1 x_2 \ldots x_{2^m-1} \colon x \in \mathscr{R}(r,m)\}$$
то есть, получается из $\mathscr{R}(r,m)$ вычеркиванием элемента вектора, соответствующего $v_1 = \ldots = v_m = 0$
\end{definition}

Очевидно, что $\mathscr{R}^*(r,m)$ имеет длину $2^m-1$, минимальное расстояние $2^{m-r}-1$.

\begin{proposition}
Для $r < m$ верно $dim(\mathscr{R}^*(r,m)) = \sum\limits_{k=0}^r C_m^k$
\end{proposition}
\begin{proof}
То есть, нужно доказать, что размерность равна размерности кода до выкалывания.
Заметим, что по лемме о рандомизации, в каждой строке порождающей матрицы четное 
количество единиц (так как $r<m$ и строки, соответствующей $v_1 \cdot \ldots \cdot v_m$,
где только одна единица, в матрице нет).

Тогда сложим все столбцы, кроме первого, и получим столбец вида $(1,0,\ldots,0)^{T}$.

Таким образом, размерность линейной оболочки всех столбцов $\mathscr{R}(r,m)$ равна размерности
линейной оболочки всех столбцов, кроме первого, то есть $dim(\mathscr{R}^*(r,m))$
\end{proof}

\subsection{Декодирование кода $\mathscr{R}(1,m)$}

Рассмотрим на примере $m=3$. Порождающая матрица $\mathscr{R}(1,m)$ будет иметь размер $(m+1) \times 2^m$
и будет состоять из векторов $\mathbf{1}, \mathbf{1}_{x_1}, \mathbf{1}_{x_2}, \mathbf{1}_{x_3}$:

$$G = \begin{pmatrix}1 & 1 & 1 & 1 & 1 & 1 & 1 & 1  \\
0 & 1 & 0 & 1 & 0 & 1 & 0 & 1  \\
0 & 0 & 1 & 1 & 0 & 0 & 1 & 1  \\
0 & 0 & 0 & 0 & 1 & 1 & 1 & 1  \\
\end{pmatrix}$$

Кодирование, как и в любом линейном коде --- домножение кодируемой строки на порождающую матрицу.

Все возможные $16$ кодов получаются линейными комбинациями строк матрицы:
$$A = \mathscr{R}(1,3) = \begin{pmatrix}
0 & 0 & 0 & 0 & 0 & 0 & 0 & 0  \\
1 & 1 & 1 & 1 & 1 & 1 & 1 & 1  \\
0 & 1 & 0 & 1 & 0 & 1 & 0 & 1  \\
1 & 0 & 1 & 0 & 1 & 0 & 1 & 0  \\
0 & 0 & 1 & 1 & 0 & 0 & 1 & 1  \\
1 & 1 & 0 & 0 & 1 & 1 & 0 & 0  \\
0 & 1 & 1 & 0 & 0 & 1 & 1 & 0  \\
1 & 0 & 0 & 1 & 1 & 0 & 0 & 1  \\
0 & 0 & 0 & 0 & 1 & 1 & 1 & 1  \\
1 & 1 & 1 & 1 & 0 & 0 & 0 & 0  \\
0 & 1 & 0 & 1 & 1 & 0 & 1 & 0  \\
1 & 0 & 1 & 0 & 0 & 1 & 0 & 1  \\
0 & 0 & 1 & 1 & 1 & 1 & 0 & 0  \\
1 & 1 & 0 & 0 & 0 & 0 & 1 & 1  \\
0 & 1 & 1 & 0 & 1 & 0 & 0 & 1  \\
1 & 0 & 0 & 1 & 0 & 1 & 1 & 0  \\
\end{pmatrix}$$

Закодируем какое-нибудь слово. Например $y = (1,0,0,1)$. $z = y G = (1, 1, 1, 1, 0, 0, 0, 0)$. 
Кодовое расстояние равно $2^{3-1} = 4$. Поэтому код способен исправить только одну ошибку.
Тогда пусть $z' = (0, 1, 1, 1, 0, 0, 0, 0)$. Преобразуем матрицу $A$ так, чтобы можно посчитать
расстояния от декодируемого вектора до всех кодовых слов. $H_{ij} = 2 A_{ij} - 1$. Можно заметить,
что $H$ --- это две матрицы Адамара, поставленные друг на друга. Преобразуем $z'$ тем же образом:
$z''_i = 2 z'_i - 1$. 

Рассмотрим $H (z'')^T$. $i$-й компонент этого вектора равен $(2^m - d(H_i, z')) - d(H_i, z') = 2^m - 2 d(H_i, z')$.
Тогда вектор с минимальным расстоянием соответствует максимуму среди компонент $H (z'')^T$.

$$H = \begin{pmatrix}
-1 & -1 & -1 & -1 & -1 & -1 & -1 & -1\\
1 & 1 & 1 & 1 & 1 & 1 & 1 & 1\\
-1 & 1 & -1 & 1 & -1 & 1 & -1 & 1\\
1 & -1 & 1 & -1 & 1 & -1 & 1 & -1\\
-1 & -1 & 1 & 1 & -1 & -1 & 1 & 1\\
1 & 1 & -1 & -1 & 1 & 1 & -1 & -1\\
-1 & 1 & 1 & -1 & -1 & 1 & 1 & -1\\
1 & -1 & -1
 & 1 & 1 & -1 & -1 & 1\\
-1 & -1 & -1 & -1 & 1 & 1 & 1 & 1\\
1 & 1 & 1 & 1 & -1 & -1 & -1 & -1\\
-1 & 1 & -1 & 1 & 1 & -1 & 1 & -1\\
1 & -1 & 1 & -1 & -1 & 1 & -1 & 1\\
-1 & -1 & 1 & 1 & 1 & 1 & -1 & -1\\
1 & 1 & -1 & -1 & -1 & -1 & 1 & 1\\
-1 & 1 & 1 & -1 & 1 & -1 & -1 & 1\\
1 & -1 & -1 & 1 & -1 & 1 & 1 & -1\\
\end{pmatrix}$$

Тогда
$$H (z'')^T = (2, -2, 2, -2, 2, -2, 2, -2, -6, 6, 2, -2, 2, -2, 2, -2)^T$$

Максимальная компонента соответсвтует 10-й строке матрицы, а это и есть вектор $(1,1,1,1,0,0,0,0)$,
который мы шифровали.


\section{Преобразование Фурье и Уолша-Адамара для булевых функций}

\begin{definition}
\emph{Экспонента }булевой функции$\exp f\left(x\right)=\left(-1\right)^{f\left(x\right)}$
\end{definition}
Т.е. $\exp f:V_{n}\stackrel{f}{\longrightarrow}\left\{ 0,1\right\} 
\longrightarrow\left\{ -1,1\right\} $


\begin{definition}
\emph{Дискретная функция Уолша}
\[
v\left(a,x\right)\coloneqq\left(-1\right)^{\left\langle a,x\right\rangle },\; a,x\in V_{n}
\]


\[
v:\tw^n\times \tw^n\longrightarrow\left\{ 1,-1\right\} 
\]

\end{definition}
На $a$ и $x$ мы смотрим одновременно и как на двоичные векторы из
$\tw^n$, и как на целые числа, двоичная запись которых, дополненная
при необходимости слева нулями, совпадает с этими векторами.


\paragraph*{Свойства функции Уолша:}
\begin{enumerate}
\item $v\left(a,x\right)=v\left(x,a\right)$
\item $\left|v\left(a,x\right)\right|=1$
\item $v\left(0,x\right)=v\left(x,0\right)=1$
\item $E$ линейное подпространство $\tw^n$\\
$a\not\in E^{\perp}$\\
Тогда 
\[
\sum_{x\in E}v\left(a,x\right)=0
\]


\begin{proof}
$E_{0}\coloneqq\left\{ x\in E:\left\langle a,x\right\rangle =0\right\} ,E_{1}\coloneqq\left\{ x\in E:\left\langle a,x\right\rangle =1\right\} $

Покажем, что $\left|E_{0}\right|=\left|E_{1}\right|$, тогда число
+1 и -1 в сумме будет одинаково.

$a\not\in E^{\perp}\Rightarrow\exists x\in E:\left\langle a,x\right\rangle \not=0\Rightarrow E_{1}\not=\emptyset$
\textcolor{blue}{$\left(E^{\perp}=\left\{ u\in \tw^n:\left\langle u,x\right\rangle =0,\forall x\in E\right\} \right)$}

Пусть $y\in E_{1}$. Рассмотрим равенство $x+y=z$. Скалярно домножив
на $a$ получим
\[
\left\langle a,x\right\rangle +\underbrace{\left\langle a,y\right\rangle }_{1}=\left\langle a,z\right\rangle 
\]
То есть если $x\in E_{1}$, то $z\in E_{0}$. И наоборот, если $x\in E_{0}$,
то $z\in E_{1}.$ Отсюда легко видеть, что, прибавляя $y$ ко всем
элементам $E_{1}$, получим элементы $E_{0}$. Значит, $E_{1}+y\subset E_{0}\Rightarrow\left|E_{1}\right|\leq\left|E_{0}\right|$.
Так же прибавляя $y$ ко всем элементам $E_{0}$, получим элементы
$E_{1}$. Значит, $E_{0}+y\subset E_{1}\Rightarrow\left|E_{0}\right|\leq\left|E_{1}\right|$.
\end{proof}
\end{enumerate}
\begin{corollary}
Т.к. ${\tw^n}^{\perp}=\left\{ 0\right\} $, 
\[
\sum_{x\in \tw^n}v\left(a,x\right)=\delta_{0}\left(a\right)2^{n}
\]
где 
\[
\delta_{0}\left(a\right)=\begin{cases}
1, & \text{если }a=0\\
0, & \text{если }a\not=0
\end{cases}
\]
\end{corollary}


\begin{definition}
\emph{Преобразованием Фурье} булевой функции $f$ называется целочисленная
функция на $\tw^n$, определяемая следующим равенством
\[
F_{f}\left(u\right)=\sum_{x\in \tw^n}f\left(x\right)v\left(x,u\right)
\]


Для каждого $u\in \tw^n$ значение $F_{f}\left(u\right)$ называется
\emph{коэффициентом Фурье}.
\end{definition}

\begin{definition}
\emph{Преобразованием Уолша-Адамара} булевой функции $f$ называется
целочисленная функция на $\tw^n$, определяемая следующим равенством
\[
W_{f}\left(u\right)=F_{f}\left(\exp f\left(u\right)\right)=\sum_{x\in \tw^n}\exp f\left(x\right)v\left(x,u\right)=
\]
\[
=\sum_{x\in \tw^n}\left(-1\right)^{f\left(x\right)}\left(-1\right)^{\left\langle x,u\right\rangle }=\sum_{x\in \tw^n}\left(-1\right)^{f\left(x\right)\oplus\left\langle x,u\right\rangle }=
\]
\[
=\sum_{x\in \tw^n}\exp\left(f\left(x\right)\oplus\left\langle x,u\right\rangle \right)
\]


Для каждого $u\in \tw^n$ значение $W_{f}\left(u\right)$ называется
\emph{коэффициентом Уолша-Адамара}.\end{definition}

Уолш: от функции Уолша.

Адамар: функцию Уолша можно получить из матрицы Адамара. Рекурсивно
умеем формировать матрицы Адамара размера $2^{n}$ (мы полученные
таким спобом матрицы матрицами Сильвестра). 
\[
H_{new}=\left[\begin{array}{cc}
H & H\\
H & -H
\end{array}\right]
\]
Тогда строчки --- функции Уолша. То есть $x$ соответствует номеру
столбца, $a$ соответствует номеру строки. Элемент $H_{a,x}=v\left(a,x\right)$.
\begin{example}
\[
H_{4}=\left[\begin{array}{ccccc}
 & \underline{x_{1}=00} & \underline{x_{2}=01} & \underline{x_{3}=10} & \underline{x_{4}=11}\\
a_{1}=00| & 1 & 1 & 1 & 1\\
a_{2}=01| & 1 & -1 & 1 & -1\\
a_{3}=10| & 1 & 1 & -1 & -1\\
a_{4}=11| & 1 & -1 & -1 & 1
\end{array}\right]
\]


\end{example}

\begin{definition}
Часто коэффициенты Фурье и коэффициенты Уолша-Адамара называются \emph{спектральными
коэффициентами}.\end{definition}
\begin{theorem}
Коэффициенты Фурье и Уолша-Адамара связаны соотношением
\[
W_{f}\left(u\right)=2^{n}\delta_{0}\left(u\right)-2F_{f}\left(u\right)
\]
\end{theorem}
\begin{proof}
\[
W_{f}\left(u\right)=\sum_{x\in \tw^n}\exp f\left(x\right)v\left(x,u\right)=\sum_{x\in\supp f}\underbrace{\exp\underbrace{f\left(x\right)}_{1}}_{-1}v\left(x,u\right)+\sum_{x\in \tw^n\setminus\supp f}\underbrace{\exp\underbrace{f\left(x\right)}_{0}}_{1}v\left(x,u\right)=
\]
\[
=-\sum_{x\in\supp f}v\left(x,u\right)+\sum_{x\in \tw^n\setminus\supp f}v\left(x,u\right)
\]
По замечанию к 4-ому свойству 
\[
\sum_{x\in \tw^n}v\left(a,x\right)=\delta_{0}\left(a\right)2^{n}
\]
\[
\sum_{x\in \tw^n\setminus\supp f}v\left(x,u\right)=\underbrace{\sum_{x\in \tw^n}v\left(x,u\right)}_{=\delta_{0}\left(u\right)2^{n}}-\sum_{x\in\supp f}v\left(x,u\right)
\]
Кроме того
\[
F_{f}\left(u\right)=\sum_{x\in \tw^n}f\left(x\right)v\left(u,x\right)=\sum_{x\in\supp f}v\left(u,x\right)
\]
Итого
\[
W_{f}\left(u\right)=\delta_{0}\left(u\right)2^{n}-2\sum_{x\in\supp f}v\left(x,u\right)=\delta_{0}\left(u\right)2^{n}-2F_{f}\left(u\right)
\]
\end{proof}
\begin{theorem}[формула обращения]
Для преобразования Уолша-Адамара справедлива формула обращения.
\[
\exp f\left(x\right)=2^{-n}\sum_{u\in \tw^n}W_{f}\left(u\right)v\left(x,u\right)
\]
\end{theorem}
\begin{proof}
\[
2^{-n}\sum_{u\in \tw^n}W_{f}\left(u\right)v\left(x,u\right)=2^{-n}\sum_{u\in \tw^n}\sum_{y\in \tw^n}\left(-1\right)^{f\left(y\right)}\underbrace{\left(-1\right)^{\left\langle y,u\right\rangle }\left(-1\right)^{\left\langle x,u\right\rangle }}_{\left(-1\right)^{\left\langle x\oplus y,u\right\rangle }=v\left(x\oplus y,u\right)}=
\]
\[
=2^{-n}\sum_{y\in \tw^n}\left(-1\right)^{f\left(y\right)}\sum_{u\in \tw^n}v\left(x\oplus y,u\right){\color{red}=}
\]
\[
\sum_{u\in \tw^n}v\left(x\oplus y,u\right)=\begin{cases}
2^{n}, & x\oplus y=0\\
0, & x\oplus y\not=0
\end{cases}
\]
Т.е. от всех сумм останется только одно слагаемое при $y=x$
\[
{\color{red}=}2^{-n}\left(-1\right)^{f\left(x\right)}2^{n}=\exp f\left(x\right)
\]

\end{proof}
Таким образом, коэффициенты Уолша-Адамара однозначно определяют булеву
функцию. Вместе с тем, не любой набор из $2^{n}$ чисел может быть
набором коэффициентов Уолша-Адамара некоторой булевой функции. 

\textcolor{blue}{Задачи 2.38, 2.39.}
\begin{theorem}[равенство Парсеваля]
Коэффициенты Уолша-Адамара удовлетворяют соотношению:
\[
\sum_{u\in \tw^n}W_{f}^{2}\left(u\right)=2^{2n}
\]
\end{theorem}
\begin{proof}
\[
\sum_{u\in \tw^n}W_{f}^{2}\left(u\right)=\sum_{u\in \tw^n}\left(\sum_{x\in \tw^n}\left(-1\right)^{f\left(x\right)\oplus\left\langle x,u\right\rangle }\right)^{2}=\sum_{u\in \tw^n}\left(\sum_{x\in \tw^n}\left(-1\right)^{f\left(x\right)\oplus\left\langle x,u\right\rangle }\right)\left(\sum_{y\in \tw^n}\left(-1\right)^{f\left(y\right)\oplus\left\langle y,u\right\rangle }\right)=
\]
\[
=\sum_{u\in \tw^n}\sum_{x,y\in \tw^n}\left(-1\right)^{f\left(x\right)\oplus\left\langle x,u\right\rangle \oplus f\left(y\right)\oplus\left\langle y,u\right\rangle }=\sum_{x,y\in \tw^n}\left(-1\right)^{f\left(x\right)\oplus f\left(y\right)}\underbrace{\sum_{u\in \tw^n}\left(-1\right)^{\left\langle x\oplus y,u\right\rangle }}_{=0\text{, при }x\not=y}=\sum_{x\in \tw^n}\sum_{x\in \tw^n}\left(-1\right)^{\overbrace{f\left(x\right)\oplus f\left(x\right)}^{0}}=
\]
\[
=2^{2n}
\]
\end{proof}
\begin{proposition}
Коэффициенты алгебраический нормальной формы $g_{f}\left(u\right)$
выражаются через коэффициенты Уолша-Адамара следующим образом:
\[
g_{f}\left(u\right)=2^{wt\left(u\right)-1}-2^{wt\left(u\right)-n-1}\sum_{\alpha\preccurlyeq u\oplus1}W_{f}\left(\alpha\right)
\]
\end{proposition}

\begin{proof}
Подставим формулу для коэффициентов Уолша-Адамара:
\[
g_f(u) = 2^{wt(u) - 1} - 2^{wt(u) - n - 1}
 \sum\limits_{\alpha \le u \oplus \mathbf{1}} 
     \sum\limits_{x \in \tw^n} \exp(f(x) \oplus \langle x, \alpha \rangle) 
 = 2^{wt(u) - 1} - 2^{wt(u) - n - 1}
   \sum\limits_{x \in \tw^n} \exp ( f(x) ) 
   \sum\limits_{\alpha \le u \oplus \mathbf{1}} \exp (\langle x, \alpha \rangle)
 \]
 Величина $\sum\limits_{\alpha \le u \oplus \mathbf{1}} \exp (\langle x, \alpha \rangle)$
 обращается в $0$ при $x$ не ортогональном $\{y \le u \oplus \mathbf{1}\}$ и равна $2^{n - wt(u)}$ при 
 $x \bot \{y \le u \oplus \mathbf{1}\} \iff x \in \{y \le u\}$.
 \[
 = 2^{wt(u) - 1} - 2^{wt(u) - n - 1}
   \sum\limits_{x \le u} \exp(f(x)) 2^{n - wt(u)} = 
   {1 \over 2} 2^{wt(u)} - {1 \over 2} \sum\limits_{x \le u} \exp(f(x)) =
   {1 \over 2}  \sum\limits_{x \le u} \underbrace{(1 - \exp(f(x)))}_{= 2 f(x)} = 
    \sum\limits_{x \le u} f(x)
   \]
\end{proof}

\section{Быстрое вычисление коэффициентов Уолша-Адамара}

Коэффициенты Уолша-Адамара "--- это коэффициенты Фурье для функции
$\exp f$, поэтому достаточно научиться вычислять коэффициенты Фурье.
Будем действовать аналогично вычислению преобразования Мёбиуса.
Пусть $u \in \tw^n$. Обозначим $u' = (u_2, \ldots, u_n)$, аналогично для
$x \in \tw^n$ обозначим $x' = (x_2, \ldots, x_n)$
$$F_f (u) = \sum\limits_{x \in \tw^n} f(x_1, x') \exp (x_1 v_1 + \langle x', u' \rangle)
 = \sum\limits_{x \in \tw^{n-1}} f(0, x) \exp(\langle x, u' \rangle)
 + f(1, x) \exp(u_1) \exp(\langle x, u' \rangle)$$
Пусть $f_0(x) = f(0, x)$ и $f_1(x) = f(1,x)$, тогда
$$F_f(u) = F_{f_0} (u') + \exp{u_1} F_{f_1} (u')$$
Таким образом, мы свели задачу преобразования Фурье для вектора
$\in \tw^{2^n}$ к задаче вычисления преобразования Фурье для двух 
векторов из $\tw^{2^{n-1}}$. Аналогично вычислению преобразования
Мёбиуса, имеем время работы $T(n) = 2^n + 2T(n-1) = 2^n \cdot n$

\section{Производная булевой функции по направлению}

\begin{definition}
Производной булевой функции $f$ по направлению $u \in \tw^n$ называется
булева функция $$D_u f (x) = f(x \oplus u) \oplus f(x)$$
\end{definition}

\begin{definition}
Производной булевой функции $f$ по направлению подпространства $L \subset \tw^n$
называется функция
$$D_u f (x) = \bigoplus\limits_{u \in F} f(x \oplus u)$$

\end{definition}

\begin{remark}
Если $L = \langle u_1, \ldots, u_k \rangle$, то $D_L f = D_{u_k} D_{u_{k-1}} \ldots D_{u_1} f$
\end{remark}
\begin{proof}
Индукция по $k$. Для $k=1$ очевидно. Пусть $L = Lin (u, L')$,
докажем, что $D_L f (x) = D_u D_{L'} (x)$.
\[
D_u D_{L'} (x) = D_u \bigoplus\limits_{v \in L'} f(x \oplus v) = 
\bigoplus\limits_{v \in L'} f(x \oplus v) \oplus \bigoplus\limits_{v \in L'} f(x \oplus v \oplus u) 
= \bigoplus\limits_{v \in L} f(x \oplus v)
\]
\end{proof}

\begin{proposition}
Верны следующие утверждения
\begin{enumerate}
\item $\forall L \subset \tw^n$ подпространства 
$\forall f \in \map(\tw^n, \tw), \, u \in L, \, x \in \tw^n$ верно
$$D_L f(x) = D_L f(x \oplus u)$$
\item $\forall f,g \in \map(\tw^n, \tw)$, $\forall$ подпространства $L \subset \tw^n$
    $$D_L (f \oplus g) = D_L f \oplus D_L g$$
\item $\forall f \in \map(\tw^n, \tw)$, $\forall u,v,x \in \tw^n$

    $$D_{u \oplus v} f(x) = D_u f(x) \oplus D_v f(x \oplus u)$$
\item $\forall u,x \in \tw^n \, D_u f(x) = 0 \iff f = \text{const}$
\item $\forall u \in \tw^n \, D_u f = \text{const} \iff 
      \exists \alpha \in \tw^n, \beta \in \tw \colon 
      f = \langle \alpha, x \rangle \oplus \beta$
      то есть $f$ --- афинная.
\end{enumerate}

\begin{proof}
\begin{enumerate}
\item $D_L f(x) = \bigoplus\limits_{v \in L} f(x \oplus v) = 
                  \bigoplus\limits_{v \in L} f(x \oplus (u\oplus v))
                  = \bigoplus\limits_{v \in L} f((x \oplus u )\oplus v)
                  = D_L f(x \oplus u)$
\item $D_L (f \oplus g) (x) = \bigoplus\limits_{u \in L} f(x \oplus u) 
          \oplus g(x \oplus v) = D_L f (x) \oplus D_L g(x)$
\item $D_{u} f(x) \oplus D_v f(x \oplus u) = f(u \oplus x) \oplus f(x)
       \oplus f(x \oplus u \oplus v) \oplus f(x \oplus u) =
        f(x) \oplus f(x \oplus u \oplus v) =  D_{u \oplus v} f(x)$
\item Очевидно по определению
\item $D_u f (x) = \alpha_u \implies f(x) \oplus f(x \oplus u) = \alpha_u$
      Тогда $f(u) = f(0) + \alpha_u$, тогда $f(x) \oplus f(y) = \alpha_{x \oplus y}
       = f(0) \oplus f(x \oplus y)$.
      Тогда $\sum\limits_{i=1}^k f(x_i) = f(\sum\limits_{i=1}^k x_i) + k f(0)$.
      Пусть $e_1, \ldots, e_n$ "--- единичные векторы. Тогда
      $$f(x) = \langle (f(e_1), \ldots, f(e_n)), x \rangle \oplus (w(x) \mod 2 \oplus 1) f(0) =
       \langle (f(e_1), \ldots, f(e_n)) + (f(0), \ldots, f(0)), x \rangle 
       \oplus f(0)$$
\end{enumerate}
\end{proof}


\end{proposition}
