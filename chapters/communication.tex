\lecture{Связь коммуникационной сложности с древовидной сложностью секущих плоскостей}{A. Рязанов}{А. Кноп}


\begin{definition}
Пусть $\phi = \bigwedge C_i$ --- формула в $k-CNF$. Тогда доказательством 
невыполнимости $\phi$ в системе Cutting Planes ($CP$) называется последовательность 
$I_1 \ge 0, I_2 \ge 0, \ldots, I_m \ge 0$
где $I_m = -1$ и
$$I_j = \sum\limits_{i=1}^{k_j} a_{ji} x_i + a_j$$

для $1 \le j < m$, $a_{ji}, a_j \in \mathbb{R}$, если каждое из $I_j$ получено
 по одному из следующих правил

\begin{enumerate}
\item ${I_1 \ge 0 \,\,\, I_2 \ge 0 \over I_1 + I_2 \ge 0}$
\item ${I \ge 0 \over aI \ge 0}$ где $a \in \mathbb{R},\, a \ge 0$
\item ${aI \ge b \over I \ge \lceil {b \over a} \rceil}$
\item ${ \over A(C_i) - 1 \ge 0}$, где $A(C)$ --- линейная форма $C$, то есть 
$A(\bigvee x_i \lor \bigvee \lnot y_i) = \sum x_i + \sum (1-y_i)$
\item ${\over x_i \ge 0}$
\item ${\over x_i \le 1}$
\end{enumerate}

Длиной доказательства в $CP$ называется сумма длин записи коэффициентов $I_j$ длиной
доказательства в $CP^*$ называется сумма модулей коэффициентов $I_j$
\end{definition}
\begin{theorem}
 Существует доказательство $PHP_{n}^{n+1}$ в $CP^*$ имеющее полиномиальный размер
\end{theorem}
\begin{proof}

$A(\lnot p_{ji} \lor \lnot p_{ki}) = -p_{ji} - p_{ki} + 2$
то есть, из малых клозов можно напрямую получить неравенства $p_{ji} + p_ki \le 1$

Из больших клозов можно напрямую получить неравенства $\sum\limits_{j=1}^n p_{ij} \ge 1$.

Тогда, если мы сможем для всех $j \in [n]$ вывести неравенство 
$\sum\limits_{i=1}^{n+1} p_{ij} \le 1$ то вывод протеворечивого равенства прост:
просуммируем выведенное по всем $j$ и получим

$$\sum\limits_{\substack{i\in[n+1] \\ j \in [n]}} p_{ij} \le n$$

С другой стороны, из больших клозов суммированием можно вывести
$$\sum\limits_{\substack{i\in[n+1] \\ j \in [n]}} p_{ij} \ge n+1$$


Далее простым вычитанием получаем противоречие.

Теперь покажем, как вывести $\sum\limits_{i=1}^{n+1} p_{ij} \le 1$. 
По индукции будем строить вывод $\sum\limits_{i=1}^{l} p_{ij} \le 1$

{\bfseries База:} $p_{1j} \le 1$, выводится сразу по аксиоме.

{\bfseries Переход:} Пусть вывели для $l$, выведем для $l+1$. Просуммируем
неравенства, соответствующие клозам $\lnot p_{l+1, k} \lor \lnot p_{j, k}$ 
для $j \le l$, получим $$l \cdot p_{l+1,k} + \sum\limits_{i=1}^l p_{ik} \le l$$
По индукционному предположению имеем $\sum\limits_{i=1}^l p_{ik} \le 1$, умножим
это неравенство на $l-1$ и прибавим к предыдущему, получая
  $$l p_{l+1, k} + \sum\limits_{i=1}^l p_{ik} \le 2l - 1$$
Теперь остается только воспользоваться правилом деления и получить
$$p_{l+1,k} + \sum\limits_{i=1}^l p_{ik} \le \Bigg\lfloor {2l - 1 \over l} \Bigg\rfloor = 1$$
что завершает доказательство.
\end{proof}

\begin{theorem}
Если у $\phi$ есть доказательство размера $S$ в системе резолюций, то
есть и доказательство размера $O(S)$ в $CP^*$.

% Здесь не совсем понятно, что считать размером доказательства в резолюциях
% я считаю, что тут это суммарное количество литералов, иначе, O(S)
% не получится.
\end{theorem}

\begin{proof}
Пусть доказательство в резолюциях имеет вид $C_1, C_2, \ldots C_m$.
Покажем, что можно вывести в $CP^*$ неравенства $A(C_1) \ge 1, \ldots, A(C_m) \ge 1$,
добавив некоторое количество дополнительных переходов.

Рассмотрим одну резолюцию в резолюционном доказательстве.
$${C_1 \lor x \,\,\, C_2 \lor \lnot x \over C_1 \lor C_2}$$
$A(C_1) + x \ge 1$ и $A(C_2) + (1 - x) \ge 1$ уже выведены. Тогда
сложением можем вывести $A(C_1) + A(C_2) \ge 1$. Проблема в том, что некоторые
слагаемые учлись два раза, то есть $A(C_1) + A(C_2) \neq A(C_1 \lor C_2)$.
Воспользуемся аксиомами $x_i \le 1 \iff -x_i \ge -1$ и прибавим
к $A(C_1) + A(C_2) \ge 1$ неравенства $-2x_i \ge -2$ для всех $x_i \in C_1 \cap C_2$.
Тогда мы получим ровно неравенство $A(C_1 \lor C_2) \ge 1$.  
\end{proof}

\begin{definition}
Коммуникационным протоколом $\rho$ будем называть дерево, в вершинах которого
стоят функции $a: X \to \{0,1\}$; $b: Y \to \{0,1\}$.
\end{definition}
