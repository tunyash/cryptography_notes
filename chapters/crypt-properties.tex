\lecture{Криптографические свойства булевых функций}{}{И. Агафонова}

\section{Нелинейность}
\begin{definition}
Нелинейностью булевой функции $f: \{0,1\}^n \to \{0,1\}$ называется число
   $$N_f = \min\limits_{g \in \mathscr{A}_n} d(f,g)$$
 где $\mathscr{A}_n$ --- пространство афинных функций (имеющих как многочлены степень не более 1) и
  $d(f,g) = |\{x \colon f(x) \neq g(x)\}|$
\end{definition}

\begin{note}
Легко видеть, что для любой $f \in \mathscr{A}_n$ существует $u \in \{0,1\}^n$ и 
$b \in \{0,1\}$ такой, что $f(x) = (u,x) \oplus b$
\end{note}

\begin{theorem}
$$N_f = 2^{n-1} - {1 \over 2} \max\limits_{u \in \{0,1\}^n} |W_f(u)|$$
\end{theorem}
\begin{proof}
\[
W_f(u) = \sum\limits_{x \in \{0,1\}^n} \exp f(x) v(x,u) = 
\sum\limits_{x \in \{0,1\}^n} (-1)^{f(x) \oplus (u,x)} = 
\]
\[
\sum\limits_{x \in \{0,1\}^n} \exp (f\oplus (u,x))(x) \underbrace{v(x,0)}_{=1} = W_{f \oplus u} (0)
= |\{x \colon f(x)\oplus (u,x) = 0\}| - |\{x \colon f(x)\oplus (u,x) = 1\}| 
\]
\[
=  (2^n - d(f(x), (u,x))) - d(f(x), (u,x)) = 2^n - 2 d(f(x), (u,x))
\]

Выражая из этой формулы расстояние 
$$ d(f(x), (u,x)) = 2^{n-1} - {1 \over 2} W_f(u)$$
теперь выразим расстояние до функции $(u,x) \oplus 1$
$$d(f(x), (u,x) \oplus 1)) = 2^n - (2^{n-1} - {1 \over 2} W_f(u)) = 2^{n-1} + {1 \over 2} W_f(u)$$
Тогда 
$$\min \{d(f(x), (u,x)), d(f(x), (u,x) \oplus 1)\} = 2^{n-1} - {1 \over 2} |W_f(u)|$$
и, наконец 
$$N_f = \min\limits_{g \in \mathscr{A}_n} d(f,g)
 = \min\limits_{\substack{u \in \{0,1\}^n \\ b \in \{0,1\}}} d(f,(u,x)\oplus b) = 
 \min\limits_{u \in \{0,1\}^n} \big|2^{n-1} - {1 \over 2} |W_f(u)|\big| =$$
 $$ = 2^{n-1} - {1 \over 2} \max\limits_{u \in \{0,1\}^n} |W_f(u)|$$
\end{proof}

\section{Автокорреляция}

\begin{definition}
Пусть $f,g \in \mathscr{F}_n = \map(\tw^n, \tw)$. Определим функцию 
$\Delta_{f,g}: \tw^n \to \mathbb{Z}$ следующим образом:
$$\Delta_{f,g} (u) = \sum\limits_{x \in \tw^n} \exp (f(x) \oplus g(x \oplus u)$$
Назовем эту функцию \emph{функцей взаимной корреляции}.
\end{definition}

\begin{proposition}
$\forall u \in \tw^n, \forall f,g \in \mathscr{F}_n \colon 
   \Delta_{f,g} (u) = \Delta_{g,f} (u)$
\end{proposition}

\begin{definition}
Для $f \in \mathscr{F}_n$ функция $\Delta_f (u) = \Delta_{f,f} (u)$ называется
функцией автокорреляции.
\end{definition}

\begin{remark}
Автокорреляция $f$ в точке $u \in \tw^n$ равна нулевому коэффициенту
Уолша-Адамара производной $f$ по направлению $u$:
$$\Delta_f(u) = W_{D_u f} (0)$$
\end{remark}

\begin{proof}
$$W_{D_f u} (0) = \sum\limits_{x \in \tw^n} \exp (D_u f(x)) 
            = \sum\limits \exp (f(x+u) + f(x)) = \Delta_f (u)$$
\end{proof}

\begin{remark}
Не любой набор из $2^n$ чисел может быть набором значений
автокорреляции.
\end{remark}

\begin{theorem}
Пусть $f,g \in \mathscr{F}_n$. Тогда
  $$(\Delta_{f,g} (0), \ldots, \Delta_{f,g} (2^n - 1)) H_n =
     (W_f(0) \cdot W_g(0), \ldots, W_f(2^n -1) \cdot W_g(2^n-1))$$
 где $H_n$ "--- матрица Сильвестра размера $2^n \times 2^n$
\end{theorem}

\begin{proof}
Без доказательства
\end{proof}

\begin{corollary}
Пусть $f \in \mathscr{F}_n$ тогда 
$$(\Delta_f(0), \ldots, \Delta_f(2^n-1)) H_n = (W_f^2(0), \ldots, W_f^2 (2^n -1)$$
или
$$\sum\limits_{x \in \tw^n} \Delta_f(x) \exp(\langle x,u \rangle) = W_f^2(u)$$
для всех $u \in \tw^n$.
\end{corollary}

\begin{theorem}
Определим $h \in \mathscr{F}_{n+m}$ как $h(x,y) = f(x) + g(y)$, где
$f \in \mathscr{F}_n,\, g \in \mathscr{F}_m$. Тогда
   $$\forall \alpha \in \tw^n, \beta \in \tw^m\colon 
                  \Delta_h(\alpha, \beta) = \Delta_f(\alpha) \Delta_g(\beta)$$
\end{theorem}

\begin{proof}
$$\Delta_h (\alpha, \beta) = 
\sum\limits_{\substack{x \in \tw^n \\ y \in \tw^m}} 
\exp( h(x,y) \oplus h(x+\alpha, y+\beta)) = $$ 
$$\sum\limits_{\substack{x \in \tw^n \\ y \in \tw^m}}
\exp(f(x) \oplus g(y) \oplus f(x + \alpha) \oplus g(y + \beta)) = \Delta_f(\alpha) \Delta_g(\beta)$$
\end{proof}
