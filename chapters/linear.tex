\lecture{Линейные коды}{А. Рязанов}{И. Агафонова}

\section{Базовые факты, коды Адамара}
\begin{definition}
Код называется линейным, если множество кодовых слов $C$ является
линейным подпространством $\tw^n$.
\end{definition}

\begin{definition}
Весом Хэмминга $a \in \tw^n$ назовем $w(a) = \{i \colon a_i = 1\}$
\end{definition}

\begin{remark}
$d(a,b) = w(a \oplus b)$
\end{remark}

\begin{lemma}
Пусть $C$ --- линейный код. Тогда $d(C) = \min\limits_{\substack{x \in C \\ x \neq 0}} w(x)$
\end{lemma}
\begin{proof}
$d(C) = \min\limits_{a\neq b \in C} d(a,b) =\min\limits_{a\neq b \in C} w(a \oplus b) =
 \min\limits_{\substack{x \in C \\ x \neq 0}} w(x)$
\end{proof}

\begin{definition}
Пусть $C$ --- некоторый линейный код с порождающей матрицей $G$ и проверочной
матрицей $H$. Тогда дуальным к нему называется код $C^{\bot}$ с порождающей
матрицей $H$ и проверочной матрицей $G$.

Если $C$ являлся $(n,k)$-кодом, то $C^{\bot}$ будет $(n,n-k)$-кодом.
\end{definition}


\begin{theorem}
Дуальный код Хэмминга $(2^m-1, 2^m-1-m)$ является кодом Адамара с матрицей
Сильвестра.
\end{theorem}

\begin{proof}
Будем доказывать по индукции.

{\bfseries База:} $m=2$. Тогда $n = 2^m - 1 = 3$, $k=2^m-1-m = 1$.
Тогда проверочная матрица такого кода Хемминга имеет вид 
$\begin{pmatrix} 0 &1 &1 \\ 1 & 0 & 1 \end{pmatrix}$
Тогда все векторы дуального кода выглядят как:
$\begin{pmatrix} 0 & 0 & 0 \\
                 0 & 1 & 1 \\
                 1 & 0 & 1 \\
                 1 & 1 & 0 \\
 \end{pmatrix}$.
Этот код совпадает с соответствующим кодом Адамара.

{\bfseries Переход:} пусть доказано для $n = 2^{m-1} - 1$.
Пусть $\bar{H} \in \tw^{(m-1) \times 2^{m-1}}$ --- проверочная матрица для кода Хэмминга
$(2^{m-1} - 1, 2^{m-1} - 1 - (m-1))$.

Покажем, что матрица 
$$ H = \begin{pmatrix}
            0 \ldots 0 & 1 & 1 \ldots 1 \\
               \bar{H} & \mathbf{0}_{m-1} & \bar{H}
        \end{pmatrix}$$
является проверочной матрицей кода Хэмминга $(2^m-1, 2^m - 1 - m)$.
Это почти очевидно, достаточно заметить, что столбцы матрицы
различны и ее размерность $m \times (2^m-1)$ (следует из
того же свойства для $\bar{H}$ и отсутствия в $\bar{H}$ нулевого
столбца. 

По индукционному предположению матрица $\bar{H}$ порождает строки
матрицы $\mathcal{A}'$ --- усеченной бинарной матрицы Адамара размера
 $2^{m-1} \times 2^{m-1} - 1$. Тогда матрица 
 $(\bar{H} | \mathbf{0}_{m-1} | \bar{H})$ порождает строки
 матрицы $(\mathcal{A}' | \mathbf{0}_{2^{m-1}} | \mathcal{A}')$.
 
Добавим в $(\bar{H} | \mathbf{0}_{m-1} | \bar{H})$ первую строку $H_1$,
чтобы получить матрицу $H$. Тогда можно сделать вывод, что
матрица $H$ порождает все строки матрицы 
$(\mathcal{A}' | \mathbf{0}_{2^{m-1}} | \mathcal{A}')$ и
строки, полученные из них прибавлением $H_0$. Тогда в
итоге мы получим коды
$$\begin{pmatrix}
    \mathcal{A}' & \mathbf{0}_{2^{m-1}} & \mathcal{A}' \\
    \mathcal{A}' & \mathbf{1}_{2^{m-1}} & \mathbf{1} - \mathcal{A}'
   \end{pmatrix}$$

Припишем слева столбец из нулей и получим, что новая матрица
--- это в точности матрица, полученная из 
$(\mathbf{0}_{2^{m-1}} | \mathcal{A}')$ по правилу Сильвестра.
Таким образом, теорема доказана.

\end{proof}

\begin{corollary}
Код Адамара с матрицей Сильвестра является линейным.
\end{corollary}

\begin{theorem}
Пусть $C$ --- линейный код, $H$ --- его проверочная матрица.
\begin{enumerate}
\item В проверочной матрице $H$  любые $d-1$ столбцов
линейно независимы $\iff$ $d(C) \ge d$
\item Если любые $d-1$ столбцов матрицы $H$ линейно независимы и
существуют $d$ линейно зависимых столбцов, то $d(C) = d$
\end{enumerate}
\end{theorem}

\begin{proof}

$\Rightarrow$

По лемме $d(C) = \min\limits_{x \in C} w(x)$. Пусть существует
$x \in C$ такое, что $w(x) < d$. $Hx = 0$. Пусть $i_1, \ldots, i_r$
--- номера ненулевых компонент $x$ ($r < d$). Тогда
 $H_{i_1} \oplus H_{i_2} \oplus \ldots \oplus H_{i_r} = 0$,
 но это противоречит условию линейной независимости столбцов.
 
$\Leftarrow$

Если $H_{i_1} \oplus \ldots \oplus H_{i_r} = 0$, то
рассмотрим вектор $x = \{x_j\}$, 
$x_j = \begin{cases} 0 & \exists l \colon j = i_l \\
                     1 & \text{ иначе} \end{cases}$,
Для такого вектора $Hx = 0$, но $w(x) = r < d$.

Пункт 2 непосредственно следует из пункта 1.
\end{proof}

\section{Смежные классы и декодирование по синдрому}

\begin{definition}
Смежным классом группы $G$ по подгруппе $C$ называется множество вида
$$\begin{array}{rl}Cb = \{x b \colon x \in C\} & \text{ правый } \\
                  bC= \{ b x \colon x \in C\} & \text{ левый }
\end{array}$$
\end{definition}

\begin{definition}
Синдром вектора $x$ относительно линейного кода $C$ с проверочной
матрицей $H$ называется вектор $Hx$
\end{definition}

\begin{theorem} 
Пусть $x,y \in \tw^n$. Тогда $x,y \in Cz$ для некоторого $z$ $\iff$
$Hx = Hy$
\end{theorem}

\begin{proof}
$\Rightarrow$ $x = a + z, \, y = b + z$, $a,b \in C$. Тогда
   $$Hx = Ha + Hz = Hz = Hb + Hz = Hy$$

$\Leftarrow$ $Hx = Hy \implies H (x + y) = 0$, тогда $x,y \in C x$.
\end{proof}

Пусть $b \in C$, $b' = b + e$, где $e$ --- вектор ошибок. Тогда
$H b' = He$, то есть, ошибку для $b'$ нужно искать в его смежном
классе по $C$.

{\bfseries Лидер} --- это слово наименьшего веса в смежном классе.
Лидер является наиболее вероятным вектором ошибок.

\begin{proposition}
Будем полагать вектором ошибок лидера соответствующего смежного класса.
Составим матрицу $A = \{A_{ij}\}_{2^{n-k} \times 2^k}$, $A_{i,0}$ ---
лидер смежного класса $i$, $A_{0,i} \in C$ и $A_{ij} = A_{i,0} \oplus
A_{0,j}$.
\begin{enumerate}
\item Исправим все ошибки, являющиеся лидерами
\item Для любого слова $A_{ij}$ слово $A_{0,j}$ является ближайшим к
     $A_{ij}$ кодовым словом.
\end{enumerate}
\end{proposition}

\begin{proof}
\begin{enumerate}
\item Очевидно
\item $A_{ij} = A_{0,j} + A_{i,0}$. $A_{i,0}$ --- лидер. 
      $d(A_{ij}, A_{0,j}) = w(A_{i,0})$. 
      
      Рассмотрим другое кодовое слово $A_{0,j'}$.
      $$d(A_{ij}, A_{0,j'}) = w(A_{ij} \oplus A_{0,j'})$$
        $$A_{ij} \oplus A_{0,j'} = 
      A_{i,0} \oplus \underbrace{A_{0,j} \oplus A_{0,j'}}_{\in C}$$
  Тогда $A_{ij} \oplus A_{0,j'}$ лежит в смежном классе $i$, значит
  $w(A_{ij} \oplus A_{0,j'}) \ge w(A_{i,0})$, что и требовалось.
\end{enumerate}
\end{proof}


\section{Полиномиальные коды}

\begin{definition}
Установим взаимно однозначной соответствие между многочленами степени 
$< n$ и двоичными векторами из $\tw^n$. 
$$\sum\limits_{i=0}^{n-1} g_i x^i \mapsto (x_0, \ldots, x_{n-1})
 \mapsto \sum\limits_{i=0}^{n-1} g_i x^i$$.
Тогда рассмотрим некоторый многочлен $g(x)$, тогда кодовые многочлены
получаются по правилу $b(x) = a(x) g(x)$, где $deg(a(x)) < k$. Тогда,
если $deg(g(x)) = n-k$, то получается $(n,k)$ код.
\end{definition}

\begin{example}
 $(6,4)$ код, с порождающим многочленом $1 + x + x^2$
 
 \begin{tabular}{ccc}
  $0 \, 0 \, 0 \, 1$ & $\overset{x^3}{\to}$ & $0 \, 0 \, 0 \, 1 \, 1 \, 1$ \\
  $0 \, 0 \, 1 \, 0$ & $\overset{x^2}{\to}$ & $0 \, 0 \, 1 \, 1 \, 1 \, 0$ \\
  $0 \, 1 \, 0 \, 0$ & $\overset{x}{\to}$ & $0 \, 1 \, 1 \, 1 \, 0 \, 0$ \\
  $1 \, 0 \, 0 \, 0$ & $\overset{1}{\to}$ & $1 \, 1 \, 1 \, 0 \, 0 \, 0$ \\     
 \end{tabular}
\end{example}

\section{Совершенные линейные коды}

\begin{definition}
Линейный $(n,k)$-код, исправляющий $r$ ошибок называется совершенным, если для него
достигается граница Хэмминга:
$$2^{n-k} = S_r(n)$$
\end{definition}

\begin{remark}
Для нелинейных кодов граница Хэмминга имеет вид
$$K = {2^n \over S_r(n)}$$
\end{remark}

\begin{example}
$(2m+1,1)$ код. Кодовые слова $\begin{pmatrix} 0 & \ldots & 0 \\ 1 & \ldots & 1 \end{pmatrix}$.
Этот код исправляет $m$ ошибок.
$$S_m(2m+1) = \sum\limits_{i=0}^{m} C_{2m+1}^i = 
{1 \over 2} \sum\limits_{i=0}^{m} (C_{2m+1}^i + C_{2m+1}^{2m+1-i}) = 2^{2m}$$
Тогда $2^{2m+1 - 1} = 2^{2m} = S_m(2m+1)$, что и требуется по определению.
\end{example}

\begin{example}
\label{perfect_hamming}
Код Хэмминга с $n = 2^m - 1$, $k = 2^m - 1 - m$, $m \ge 2$. Код исправляет одну ошибку,
$S_1(n) = 1 + n = 2^m$. Тогда
$$2^{n-k} = 2^{2^m - 1 - (2^m - 1 - m)} = 2^m = S_1(n)$$
\end{example}

\begin{theorem}
Cледующие условия равносильны
\begin{enumerate} 
\item Существует двоичный совершенный код $C$ в $\tw^{n}$, который исправляет
одну ошибку
\item $n = 2^m - 1$ 
\end{enumerate}
\end{theorem}

\begin{proof}
$2 \implies 1$ 
Должно выполняться $K = {2^n \over n+1}$. $K$ может быть целым, только если 
$n+1 = 2^m$ для некоторого m.

$1 \implies 2$ Доказали в примере \ref{perfect_hamming}.
\end{proof}

\begin{example}
$(23,12)$-код Голея, исправляющий $3$ ошибки. $S_3(23) = 1 + 23 + C_{23}^2 + C_{23}^3 
= 2048 = 2^{11}$. Тогда $2^{23} = S_3(23) \cdot 2^{12}$.
\end{example}

\section{Двоичные циклические коды}

\begin{definition}
Линейный код $C$ называется циклическим, если $\forall b \in C \colon b^{(1)} \in C$,
где $(b_0, \ldots, b_{n-1})^{(1)} = (b_{n-1}, b_0, \ldots, b_{n-2})$
\end{definition}
